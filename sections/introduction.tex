%!TEX root = ../paper.tex
\label{introduction}

With the advent of collisions at $\sqrt{s} = 13$~TeV, the Large Hadron Collider has once again broken records at the energy frontier. Collisions at these unprecedented energy scales, and even those at the lower LHC runs at $\sqrt{s} = 7$ and $8$~TeV, are unlike those at any previous collider because they produce significant numbers of massive particles with large (transverse) momentum. As the opening angles betewen the decay products of these massive particles is inversely proportional to the transverse momentum, these decays are highly collimatted, leading to inefficiencies in \textit{classical} event reconstruction techniques.

The field of \textit{jet substructure} began as an effort to address these inefficiencies: by aiming to reconstruct massive particles in a single, large-radius jet (instead of several overlapping small-radius jets), new techniques were hoped to be able to recover and improve the understanding of these high-energy particles. \textit{Grooming} methods were designed to remove the susceptibility of the large jets to pile-up and underlying event by removing portions of the jets that were measured to be uninteresting;  \textit{tagging} methods were designed to exploit the difference in the energy distributions between signal jets from electroweak objects and background jets from QCD multi-jet production. Since its origins in improving searches for the Higgs boson and new physics, the field has expanded also to measurement of the Standard Model: the stucture of jets, exactly the information which jet substructure utilizes, can be utilized as sensitive tests of parton showers, hadronization models, and fundamental QCD properties. 

The BOOST conferences have been a series of workshops designed to bring together experts in substructure from both theoretical and experimental communities.

These proceedings are designed to serve as both a summary of the BOOST 2014 and 2015 conferences, and as a review of the status of jet substructure in general. The field has evolved significantly since the last review in the 2012 BOOST report, and our goal is to highlight these developments. %Moreover, as BOOST has always served as a forum for discussion 